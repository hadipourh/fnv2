% IACR Transactions CLASS DOCUMENTATION
% Written by Gaetan Leurent gaetan.leurent@inria.fr (2016-2018)
%
% To the extent possible under law, the author(s) have dedicated all
% copyright and related and neighboring rights to this software to the
% public domain worldwide. This software is distributed without any
% warranty.
%
% You should have received a copy of the CC0 Public Domain Dedication
% along with this software. If not, see
% <http://creativecommons.org/publicdomain/zero/1.0/>.

%%% script AUTHOR: Ga{\"{e}}tan Leurent, Friedrich Wiemer%%%
%%% script TITLE: {IACR} Transactions class documentation%%%
\documentclass{iacrtrans}
\usepackage[utf8]{inputenc}
\author{Hosein Hadipour\inst{1}}
\institute{University of Tehran, Iran, \email{hosein.hadipour@protonmail.com}}
%\title[\texttt{iacrtans} class documentation]{Some Collisions for the \texttt{FNV2}}
\title[Hosein Hadipour NSUCRYPTO-2018]{Some Collisions for the \texttt{FNV2}}
\subtitle{Yet another application of \texttt{LLL}}

\begin{document}

\maketitle

% use optional argument because the \LaTeX command breaks the PDF keywords
\keywords[\publname, ToSC, TCHES, LaTeX]{LLL algorithm, \texttt{FNV1-a}, \texttt{FNV2}, \texttt{SageMath}}

\begin{abstract}
 In this document we discuss about the solution of the problem 4 of the NSUCRYPTO 2018 competition. This problem is about a plain hash function called \texttt{FNV2}, which is derived from a real non-cryptographic hash function \texttt{FNV-1a}.
\end{abstract}

%\tableofcontents{}

\section*{Introduction}
In this document we want to find some collisions for a plain non-cryptographic hash function called \texttt{FNV2}. The \texttt{FNV2} is a simplified version of the \texttt{FNV1-a}  which uses modular addition instead of the \texttt{XOR} operation.


\section{\texttt{FNV2} hash function}
The \texttt{FNV2} hash function is derived from the hash function \texttt{FNV-1a}. \texttt{FNV2} processes a message $x$ composed of bytes $x_{1}, x_{2}, \ldots , x_{n}\in \{0, 1, \ldots, 255\}$ in the following way:
\begin{itemize}
\item 
$h\gets h_{0}$
\item 
for $i = 1, 2, \ldots, n:  \ h\gets (h + x_{i})g \mod 2^{128}$;
\item 
return h. 
\end{itemize}
Here $h_{0} = 144066263297769815596495629667062367629$, and $g = 2^{88} + 315$.

We want to find some collisions for the \texttt{FNV2}, that is, two different messages $x$ and $x'$ such that $\texttt{FNV2}(x) = \texttt{FNV2}(x')$. This is actually the problem number 4 of the second round of the NSUCRYPTO-2017 olympiad. 


\section{From the hash collision problem to the LLL algorithm}
Firstly, it is clear taht
\begin{equation*}
\texttt{FNV2}(x_{1}, x_{2}, \ldots, x_{n}) = (h_{0}g^{n} + x_{1}g^{n} + x_{2}g^{n-1} + \ldots + x_{n}g) \mod 2^{128}.
\end{equation*}
Next, it is sufficient to solve the equation
\begin{equation}
\label{collision_equation}
z_{1}g^{n - 1} + z_{2}g^{n - 2} + \ldots + z_{n}g^{0} \equiv 0 \mod 2^{128}
\end{equation}
in $z_{1}, z_{2}, \ldots, z_{n}\in \{-255,\ldots, 255\}$ not equal to zero  simultaneously. Indeed, $z_{i} = x_{i} - y_{i}$ for some $x_{i}, y_{i}\in \{0, \ldots, 2555\}$, and 
\begin{equation*}
\texttt{FNV2}(x_{1}, x_{2}, \ldots, x_{n}) - \texttt{FNV2}(y_{1}, y_{2}, \ldots, y_{n}) = g(z_{1}g^{n-1} + z_{2}g^{n-2} + \ldots + z_{n}g^{0}) \equiv 0 \mod 2^{128}. 
\end{equation*}
Since $\gcd(g, 2^{128}) = 1$, we can multiply two sides of the above equivalence by $g^{-1}$ and derive the \ref{collision_equation}. 

Therefore the purpose is to construct a polynomial such that g is its root. Note that this is not a natural integer relation problem, since the addition in the above equations is modular addition. 

Let us define integer vectors $e^{0}, \ldots, e^{n}$ of length $n + 1$ in the following way:

\begin{align*}
&
e^{0} = (\underbrace{0, \ldots, 0}_{n}, t.2^{128}), \text{where $t$ is a small integer},\\
&e^{i} = (\underbrace{0, \ldots, 0}_{i-1}, 1, \underbrace{0, \ldots, 0}_{n-i}, g^{n - i} \mod 2^{128}), \text{where} i \in \{1, \ldots, n\}.
\end{align*}
Let us add $z_{0}$ to $z_{1}, \ldots, z_{n}$ and consider the linear combination
\begin{equation*}
l_{z} = z_{0}e^{0} + \ldots + z_{n}e^{n} = (z_{1}, \ldots, z_{n}, z_{0}.t.2^{128} + z_{1}.g^{n-1} + z_{2}.g^{n-2} + \ldots + z_{n}.g^{0}).
\end{equation*}
To solve the problem it is sufficient to find a liner combination $l_{z}$ with $z_{1}, \ldots, z_{n}\in \{-255, \ldots, 255\}$ and zero last coordinate. This can be done using \texttt{LLL} algorithm. It is a lattice reduction algorithm that can find a short nearly orthogonal basis $\langle e^{0}, \ldots, e^{n} \rangle$. Obtaining such an \texttt{LLL}-reduced basis, we check if it contains a vector $l_{z}$ with desired properties\cite{gorodilova2018problems}. A \texttt{SageMath} \cite{sagemath} code which in the above solution has been implemented can be found here: \url{https://github.com/hadipourh/fnv2}.

\bibliography{references}{}
\bibliographystyle{plain}
\end{document}
